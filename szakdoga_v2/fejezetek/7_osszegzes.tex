% !TEX encoding = UTF-8 Unicode

\Chapter{Összegzés}

A dolgozat a művészi szűrők témakörét mutatta be azok matematikai modelljén és néhány saját szűrő megvalósítás segítségével.

A dolgozat megírása előtt még nem foglalkoztam sem képfeldolgozással, sem azokkal az algoritmusokkal amelyeket itt említettem és használtam. Mindig is érdekelt, hogy ezek hogyan működnek. Az első pár hónapban komoly háttérkutatásokat végeztem, mind az algoritmusok és azok matematikai hátterével kapcsolatban.

Mint az előző fejezetekben olvasható, négy saját szűrőt raktam össze amelyek, rajzfilm, ceruzarajz-szerű, továbbá festmény jellegűek. Ezek mindegyikének bemutattam a matematikáját, valamint az implementációját. Néhányat ezek közül, leírások segítségével állítottam össze, de volt olyan amelyek teljesen saját ötlet alapján készült. A filterek algoritmusainak matematikai háttere az előző háttérkutatás után már nem volt ismeretlen, így már csak egy eszköz kellett, amivel meg is lehetett valósítani ezeket. 

Korábban nem használtam az OpenCV könyvtárat. A C++ programozási nyelv használata tünt a legcélszerűbbnek. (Eleinte C-ben kezdtem a kódok írását, de több olyan algoritmus implementációja hiányzik az OpenCV könyvtárból ami C++-ban viszont megtalálható. Ezen algoritmusokat viszont fontosak voltak a saját készítésű filterekhez.) Az algoritmusok a könyvtárban könnyedén használhatók, mint ahogy a \aref{chap:implement}. fejezetben részletezem. Egyszerűen megadjuk a kívánt paramétereket és már is elértük a kívánt műveletet.

Tesztekkel meg sikerült vizsgálni, hogy a szűrők használata során az egyes lépések számítási ideje milyen. Ezáltal jól láthatóvá váltak a képfeldolgozás szempontjából költségesebb műveletek. A videókon, továbbá a valós időben való képfeldolgozás a saját készítésű szűrők esetén néhány helyen még további kutatást és fejlesztést igényelne. A videók képe vibrál, ahogy \aref{chap:tests}. fejezetben is említettem, de akár látható is a dolgozathoz csatolt CD-mellékleten, ha futtatjuk ezen alkalmazásokat. Ezen hibák javítására szerepelnek javaslatok a dolgozatban, viszont javításukhoz egy külön, teljes körű, célzottan erre a problémakörre koncentráló kutatásra lenne szükség.

\newpage

\section*{Summary}

This work presents the topic of artistic filters, their mathematical foundations, and some of my own filter implementations. 

This is my first time working with image processing algorithms. I was always interested in how they work. In the first couple of months, I did background research about algorithms and their mathematical background. I have shown the results of this research in Chapter 3. 

I have designed and implemented four filters for cartoon, pencil and painting-like filtering. These filters take both theoretical and practical aspects into consideration. I have used the available literature, but the filters are based on my original ideas. Background research was conducted, mathematical formulas were given, and appropriate software tools were found for the implementation. 

I have never used the OpenCV library before. The usage of the C++ programming language seems to be the proper solution. (Initially, I started to code in C, but some algorithms are implemented in C++, which were necessary for my filters.) The algorithms of the library are easy to use, as we can see in Chapter 5. I have provided the appropriate parameters and reached the desired operation. 

I have checked the calculation time of the filtering steps, revealing the time consuming filtering operations. Some aspects of the video and real-time image processing require further research and development. The vibrating noise in the videos (as mentioned in Chapter 6 and can be checked by running the software) should be also filtered. I have proposed some solutions for reducing this type of noise, but their detailed consideration is out of the scope of this work.
