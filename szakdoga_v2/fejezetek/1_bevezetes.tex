% !TEX encoding = UTF-8 Unicode

\Chapter{Bevezetés}

\hspace{12.5mm}Szakdolgozatom célja egy termelő- és kereskedő vállalatok számára specializálható, grafikus felhasználói felülettel rendelkező adatbáziskezelő szoftver létrehozása volt, amellyel az említett vállalkozások raktározási és logisztikai ügyeit egyszerűen, átláthatóan, megelőző informatikai ismeretek nélkül is lehessen kezelni, azon a logikán alapulva, amit korábban papíron alkalmaztak.

% elméleti jelentőség

\setlength{\parindent}{12.5mm}Az alkalmazás elméleti jelentősége az iránymutatás a vállalatok részére, hogy a jelenleg is zajló, negyedik ipari forradalom jegyében lépjenek előre, és zárkózzanak fel a digitális technológia vívmányaival karöltve a hatékonyabb és kevesebb emberi munkavégzés érdekében, a logisztika, raktározás és a valós időben történő nyomkövetés használatával, akár személyi számítógépeken, akár mobil eszközökön. Megelőző kutatásaim és személyes tapasztalataim alapján tisztában vagyok vele, hogy digitális bevándorlóként sokkal nehezebb átlátni és megérteni egy számítógépes program működését, de éppen ezért támasztottuk a programmal szemben azt az alapvető követelményt, hogy könnyen értelmezhető, felhasználóbarát kezelőfelülettel rendelkezzen, úgy, hogy egy informatikában kevésbé jártas személy is megfelelően tudja használni.

% gyakorlati jelentőség

\setlength{\parindent}{12.5mm}A szoftver gyakorlati jelentősége az előző pontban említett elméleti előrelépés alkalmazása esetén valósul meg. A digitális átállás jelentősen leegyszerűsíti és felgyorsítja a munkát, valamint a digitális munkavégzés által csökkentett papírhasználat környezetbarát is.

% jelentőség rám nézve

\setlength{\parindent}{12.5mm}Szakdolgozatom elméleti és gyakorlati jelentősége rám tekintve, hogy az eddig tanult, jellemzően fiktív adatokra épülő szoftveres megoldásokat már létező, valós problémakörre tudtam kiterjeszteni. Ehhez nagyban hozzájárult az, hogy az egyetemi alapképzés kereteiben megfelelő szintű tudást sajátítottam el ahhoz, hogy valós adatokkal dolgozó szoftvereket készítsek, amik gyakran nagy felelősséggel járnak.